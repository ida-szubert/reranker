\section*{Q3}
Table~\ref{BLEUtable} and figure~\ref{BLEU} illustrate the BLEU scores achieved by rerankers using a variety of feature combinations.

\begin{figure}
	\centering
	\includegraphics[scale=.75]{figures/q3.pdf}
	\caption{BLEU score as dependent on candidate translation features used for reranking.}\label{BLEU}
\end{figure}

\begin{table}
	\centering
	\begin{tabular}{||c|c|c|c||}
		\hline
		\textbf{p(e)} & \textbf{p(f$|$e)} & \textbf{p\_lex(e$|$f)} & \textbf{BLEU}\\
		\hline
		1&1&1&27.35\\
		\hline
		\multicolumn{4}{||c||}{manipulating p(e)}\\
		\hline
		1&0&0&26.29\\
		0&1&1&26.21\\
		-1&1&1&24.63\\
		-1&0&0&24.19\\
		\hline
		\multicolumn{4}{||c||}{manipulating p(f$|$e)}\\
		\hline
		0&1&0&26.63\\
		1&0&1&26.70\\
		1&-1&1&26.21\\
		0&-1&0&25.58\\
		\hline
		\multicolumn{4}{||c||}{manipulating p\_lex(e$|$f)}\\
		\hline
		0&0&1&25.77\\
		1&1&0&26.43\\
		1&1&-1&25.50\\
		0&0&-1&24.84\\
		\hline
		\end{tabular}
		\caption{BLUE scores achieved with different combinations of feature weights.}\label{BLEUtable}
\end{table}

The first thing to note is that the best BLEU score model is achieved when all three features are used, which suggests that all of them contribute useful information.

The influence of particular features on the BLEU score is quite opaque. There is no single most informative feature. The best one to use in isolation is p(f$|$e); the largest BLEU decrease is observed when p(e) is eliminated or reversed. Interestingly, combining TM likelihood with one other feature has a negative influence on the score, but combining it with both brings the score to maximum.  
