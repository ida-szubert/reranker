\section*{Q1}
The default limitation of both stack size and maximum number of translation
to 1 is very restrictive and so one expects to see a significant improvement
in total log probability once either limit is increased. This proves to be
true, but only in a very limited range for both variables, as illustrated in
Figure~\ref{decode1}.
\vspace{4mm} %4mm vertical space

Changing the maximum number of translations causes a sharp improvement, but
it flattens out very quickly. The difference in total log probability
between decoder with $k=10$ and $k=50$ is only 10\% of the difference
between $k=1$ and $k=2$. No further improvements are observed for $k>50$.

Changing the stack size has a similar effect, although improvement tapers
out even quicker. Increasing stack size to over 3 does not provide
significant benefits.

Qualitatively, the difference between translations produced by the maximally
limited system ($k=1$ and $s=1$) and one which produces the highest scoring
output ($k=50$ and $s=10$) is that the latter text is more fluent and
idiomatic, yet the meaning is equally difficult to grasp and there is no
difference between the levels of syntactic correctness.

We did not investigate how parameter changes affect decoding speed. Simple
timing methods seemed to be neither sensitive nor reliable enough to uncover
any interesting trends for values of $k$ and $s$ in the range [1--100]. However,
Figure~\ref{swap} offers a decoding time comparison with the local reordering case. 
\vspace{4mm} %4mm vertical space

The following conclusions can be drawn:
\begin{enumerate}
	\item
    The intuitive insight that phrases can have multiple good translations,
    depending on the context, proves to be important. Increasing the maximum
    number of possible translations per phrase does improve output quality.
	\item
    The pruning heuristics used in decoding are justifiable, since the best
    translation can really be found by searching within the top few
    hypothesis in each stack. The best hypothesis is not always the very top
    one, as witnessed by the sharp increase in log probability when stack
    size is changed to 2, but it seems to almost always be one of the top 5
    or so.
	\item
    There is a limit, and one that's quickly reached, to how good a
    monotonic decoder can perform. Allowing it to entertain more hypotheses
    cannot overcome inherent problems entailed by monotonicity.
\end{enumerate}

\begin{figure}
	\centering
	\begin{subfigure}{.8\linewidth}
		\includegraphics[scale=.65]{figures/d1_k.pdf}
		\caption{Increasing maximum number of translations, range [1--100].}
	\end{subfigure}
	\hskip2em
	\begin{subfigure}{.8\linewidth}
		\includegraphics[scale=.65]{figures/d1_s.pdf}
		\caption{Increasing stack size, range [1--100].}
	\end{subfigure}
	\caption{Performance of decoder without reordering.}\label{decode1}
\end{figure}
